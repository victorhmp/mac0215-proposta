\documentclass[10pt,twoside,a4paper]{article}
% ^- openany - open new pages in odd/even page

%packages
\usepackage[T1]{fontenc}
\usepackage[utf8]{inputenc}      % Encoding
\usepackage[portuguese]{babel}   % Correção
%\usepackage{caption}             % Legendas
\usepackage{enumerate}

% Matemática
\usepackage{amsmath}             % Matemática
\usepackage{amsthm, amssymb}     % Matemática
\newtheorem*{def*}{Definição}
\newtheorem*{invariant}{Invariante}

% Gráficos
\usepackage[usenames,dvipsnames]{color}  % Cores
\usepackage[pdftex]{graphicx}   % usamos arquivos pdf/png como figura
\usepackage[usenames,svgnames,dvipsnames]{xcolor}

% Desenhos
\usepackage{tikz}
\usepgfmodule{decorations}
\usetikzlibrary{patterns}
\usetikzlibrary{decorations.shapes}
\usetikzlibrary{shapes.geometric}
\usetikzlibrary{decorations.text}
\usetikzlibrary{positioning} % Adjust grid size

% Código-fonte
\usepackage[noend]{algpseudocode}
\usepackage{algorithm}

% Configurações da página
\usepackage{fancyhdr}           % header & footer
\usepackage{float}
\usepackage{setspace}           % espaçamento flexível
\usepackage{indentfirst}        % Identa primeiro parágrafo
\usepackage{makeidx}
\usepackage[nottoc]{tocbibind}  % acrescentamos a  bibliografia/indice/
                                % conteudo no Table of Contents
                                
% Fontes
%\usepackage{helvet}
\renewcommand{\familydefault}{\sfdefault}
\usepackage{type1cm}            % fontes realmente escaláveis
\usepackage{titletoc}
\usepackage{pdflscape}          % Páginas em paisagem
\usepackage{pdfpages}

% Fontes e margens
\usepackage[fixlanguage]{babelbib}
\usepackage[font=small,format=plain,labelfont=bf,up,textfont=it,up]{caption}
\usepackage[a4paper,top=3.0cm,bottom=3.0cm,left=2.0cm,right=2.0cm]{geometry}

% Referências e citações
\usepackage[
    pdftex,
    breaklinks,
    plainpages=false,
    pdfpagelabels,
    pagebackref,
    colorlinks=true,
    citecolor=DarkGreen,
    linkcolor=DarkBlue,
    urlcolor=DarkRed,
    filecolor=green,
    bookmarksopen=true
]{hyperref} 

\usepackage{hyperref}
\usepackage[all]{hypcap} % Soluciona o problema com o hyperref e capitulos
% \usepackage[round,sort,nonamebreak]{natbib} % Citação bibliográfica plainnat-ime
%\bibpunct{(}{)}{;}{a}{\hspace{-0.7ex},}{,}  % Estilo de citação
% \bibpunct{(}{)}{;}{a}{,}{,}

% Info
\title{MAC0215 - Proposta}
\author{
  \begin{tabular}{rl}
    Aluno:      & Victor Hugo Miranda Pinto \\
    Supervisor: & Alfredo Goldman vel Lejbman \\
    Co-supervisor: & Renato Cordeiro Ferreira
  \end{tabular}
}
\date{2º Semestre de 2018}

\graphicspath{ {./img/} }
 
%%%%%%%%%%%%%%%%%%%%%%%%%%%%%%%%%%%%%%%%%%%%%%%%%%%%%%%%%%%%%%%%%%%%%%%%
\begin{document}

\onehalfspacing

\maketitle
%%%%%%%%%%%%%%%%%%%%%%%%%%%%%%%%%%%%%%%%%%%%%%%%%%%%%%%%%%%%%%%%%%%%%%%%

\section{Resumo}
%%%%%%%%%%%%%%%%%%%%%%%%%%%%%%%%%%%%%%%%%%%%%%%%%%%%%%%%%%%%%%%%%%%%%%%%
  Hackathons ("hacker marathons") são eventos em que os participantes desenvolvem protótipos de software ou hardware num tempo restrito. O HackathonUSP é um hackathon universitário que tem como objetivo ser acessível para novos alunos e estimular a inovação tecnológica na Universidade de São Paulo (USP). Este projeto propõe a criação de um algoritmo para seleção automática de participantes do evento, utilizando dados dos inscritos e o número de participantes desejado. O algoritmo será construído usando um modelo de florestas aleatórias cujos critérios de decisão seguirão os utilizados pelos organizadores em seleções manuais e cujo treinamento será feito com dados coletados nas edições anteriores do evento. Dessa maneira, espera-se tornar a seleção mais rápida, transparente e reprodutível, beneficiando as próximas edições do HackathonUSP e criando um precedente para outros eventos do tipo.

\section{Introdução}
%%%%%%%%%%%%%%%%%%%%%%%%%%%%%%%%%%%%%%%%%%%%%%%%%%%%%%%%%%%%%%%%%%%%%%%%

  Nesta sessão será apresentada uma descrição de hackathons, dando uma definição geral sobre esse tipo de evento, e introduzindo o HackathonUSP em específico. Em seguida, será apresentada a proposta da plataforma Hacknizer, onde será integrado o código gerado nesta pesquisa.

  \subsection{Hackathons}
  %%%%%%%%%%%%%%%%%%%%%%%%%%%%%%%%%%%%%%%%%%%%%%%%%%%%%%%%%%%%%%%%%%%%%%
  
    O termo \textit{hackathon} vem da combinação das palavras \textit{hacking} e \textit{marathon}. Hackathons são eventos contínuos em que os participantes se organizam em pequenos times com o objetivo de criar um protótipo de software ou hardware sob um limite de tempo (24 horas em seu formato mais comum) \cite{Komssi2015WhatFor}.
    
    Hackathons são realizados com objetivos e de maneiras diferentes. São feitos tanto por empresas (normalmente de tecnologia) quanto por instituições de ensino, sociedade civil ou governo.
    
    Muitas empresas de tecnologia realizam hackathons internos, ou seja, apenas para seus funcionários, com o objetivo de gerar inovação dentro da empresa ou para fazer com que os desenvolvedores trabalhem em projetos diferentes dos do seu dia a dia. Dependendo dos protótipos desenvolvidos durante o evento, a empresa pode decidir implantar a ideia. Uma empresa famosa por sua cultura de hackathons internos é a \textit{Facebook}. É comum que essas empresas também realizem hackathons abertos ao público, em geral com o objetivo de promover um novo produto na comunidade desenvolvedora. Em geral, esses eventos também incluem algum tipo de premiação para motivar a participação, constituindo portanto uma competição.
    
    No contexto de instituições de ensino superior, são realizados hackathons \textit{universitários}. Esse tipo de evento possui objetivos diferentes daqueles das empresas, já que costumam não ter fins lucrativos e buscam incentivar a criatividade e a inovação no ambiente universitário. Hackathons universitários são muito efetivos como ferramenta educacional para o crescimento de futuros desenvolvedores. Esses eventos fornecem um ambiente para aprendizado e experimentação com novas tecnologias. Ao mesmo tempo incentivam os alunos a criarem novas soluções para a comunidade interna \cite{Kayastha2017EnablingCompetition}. Além disso, propiciam a integração entre alunos de diferentes cursos, promovendo a interdisciplinaridade, a troca de conhecimentos e experiências.

  \subsection{HackathonUSP}
  %%%%%%%%%%%%%%%%%%%%%%%%%%%%%%%%%%%%%%%%%%%%%%%%%%%%%%%%%%%%%%%%%%%%%%
    
    O HackathonUSP é o um dos maiores hackathons universitários do Brasil, com número crescente de inscritos e participantes. Ele tem como  público-alvo os alunos de graduação e pós da Universidade de São Paulo. É organizado pelo grupo de extensão USPCodeLab (UCL) em parceria com o Núcleo de Empreendedorismo da USP (NEU) e o Hardware Livre USP (HL).
    
    O objetivo do evento é ser um hackathon acessível, o primeiro evento do tipo para alunos da Universidade, ao passo que ajude a própria comunidade USP a gerar soluções relevantes para as suas demandas.
    
  \subsection{Seleção de participantes}
  %%%%%%%%%%%%%%%%%%%%%%%%%%%%%%%%%%%%%%%%%%%%%%%%%%%%%%%%%%%%%%%%%%%%%%
    
    Um dos elementos mais decisivos para o sucesso do HackathonUSP é a seleção dos seus participantes. Como o evento historicamente possui mais inscritos que vagas, a etapa de seleção é importante para maximizar a participatividade (i.e., evitar a desistência) e a diversidade (i.e., evitar o viés de público existente na computação) do evento, possibilitando o impacto proposto pelos organizadores.
    O algoritmo de seleção atual pode ser descrito da seguinte maneira: \\
    
    
    Sejam N o número de inscritos no evento, M o número de participantes e D a taxa de desistência entre os selecionados para o evento. Então:
    
    \begin{enumerate}
      \item Selecionar os primeiros M inscritos, por ondem de inscrição.
     
      \item Selecionar mais inscritos para obter M participantes efetivos contando com um percentual de desistência D (em geral, 150\% de M para D igual a 33\%), levando em consideração: 
      
      \begin{itemize}
          \item o completamento de times: selecionando um inscrito que participa de um time cujos demais integrantes foram selecionados nos primeiros M;
          \item a paridade de gêneros: selecionando mulheres que não tenham times para equilibrar a proporção de gênero entre os selecionados (a área da computação conta com um percentual pequeno de mulheres comparativamente ao de homens \cite{Frenkel1990WomenComputing});
          \item o desempenho em edições anteriores: como último critério de completamento, selecionar times cujos membros tiveram destaque em edições anteriores mesmo que nenhum deles estivesse entre os M primeiros inscritos.
      \end{itemize}   
    \end{enumerate}
    
  \subsection{Hacknizer}
  %%%%%%%%%%%%%%%%%%%%%%%%%%%%%%%%%%%%%%%%%%%%%%%%%%%%%%%%%%%%%%%%%%%%%%

    O Hacknizer é uma plataforma para organizar e hospedar hackathons, visando prover uma solução integrada e completa para a criação desses eventos. Ele está sendo construído pelo \textit{USPCodeLab} e seu desenvolvimento começou durante a \textit{USPCodeLab Winter School 2018}, um evento do tipo \textit{dev.camp} (parte do ciclo \textit{dev.journey}) organizados pelo grupo com objetivo de experimentar novas tecnologias disponíveis para o desenvolvimento Web.
    
    A arquitetura do sistema segue o padrão de microsserviços, ou seja, a estrutura de \textit{backend} é constituída por módulos cujo desenvolvimento, implantação e replicação são semi-independentes, permitindo melhor escalabilidade do sistema \cite{Newman2015BuildingSystems}. Os serviços existentes são:
    
    \begin{enumerate}
        \item \textbf{Hackathons}: responsável por gerenciar informações referentes a cada série de hackathons e suas respectivas edições hospedadas no sistema.
        
        \item \textbf{\textit{Auth}}: responsável por autenticação de usuários e autorização de acesso, além do armazenamento de dados básicos dos usuários (nome, e-mail, etc.).
        
        \item \textbf{Gerador de PWAs}: responsável pela criação de \textit{Progressive Web Apps} (PWAs) para cada hackathon, ou seja, páginas web instaláveis como aplicativos para cada evento.
        
         \item \textbf{Participantes}: responsável pelo gerenciamento dos usuários associados aos hackathons, incluindo seus papéis numa dada edição (organizador, participante, mentor, jurado ou convidado).
         
         \item \textbf{Grupos}: responsável pelo gerenciamento dos grupos de participantes dentro de um hackathon e pela submissão de projetos.
         
         \item \textbf{Seleção}: responsável pela seleção manual de participantes a serem convidados para um certo hackathon a partir da lista de inscritos.
         
         \item \textbf{\textit{API Gateway}}: responsável por receber requisições dos clientes da plataforma (Web e mobile) e direcionar essas requisições para os microsserviços corretos afim de construir uma única resposta para enviar de volta ao cliente.
    \end{enumerate}
    
    O projeto foi desenvolvido utilizando ferramentas e tecnologias tais como:
    
    \begin{itemize}
        \item \href{https://www.docker.com/}{\textbf{\textit{Docker}}}: usado para conteinerização dos módulos.
        \item \href{https://kubernetes.io/}{\textbf{\textit{Kubernetes}}}: usado para gerenciamento de \textit{clusters}.
        \item \href{https://graphql.org/}{\textbf{\textit{GraphQL}}}: especificação para design de APIs.
        \item \href{https://nodejs.org/en/}{\textbf{\textit{Node}}}: linguagem e ambiente de programação para \textit{backend}.
        \item \href{https://koajs.com/}{\textbf{\textit{Koa}}}: \textit{microframework} de \textit{Node} para construir servidores HTTP.
        \item \href{https://www.apollographql.com/server}{\textbf{\textit{Apollo Server}}}: biblioteca para construir servidores GraphQL.
        \item \href{https://www.postgresql.org/}{\textbf{\textit{PostgreSQL}}}: sistema gerenciador de bancos de dados relacionais para consulta.
        \item \href{https://kafka.apache.org/}{\textbf{\textit{Kafka}}}: transmissor de mensagens para consistência entre serviços.
        \item \href{https://prisma.io/}{\textbf{\textit{Prisma}}}: mapeador do esquema de APIs GraphQL para o PostgreSQL.
    \end{itemize}
    
%%%%%%%%%%%%%%%%%%%%%%%%%%%%%%%%%%%%%%%%%%%%%%%%%%%%%%%%%%%%%%%%%%%%%%%%

\section{Objetivos}
%%%%%%%%%%%%%%%%%%%%%%%%%%%%%%%%%%%%%%%%%%%%%%%%%%%%%%%%%%%%%%%%%%%%%%%%

    Os objetivos dessa pesquisa são:
    \begin{itemize}
        \item \textbf{Automatizar a seleção de participantes}\\
          Buscar uma maneira de automatizar o processo de seleção já existente para o HackathonUSP. Com isso, será possível aumentar a transparência na organização do evento, garantir a reprodutibilidade do processo e diminuir o trabalho envolvendo tal seleção dado que o evento vem crescendo a cada edição.
        
        \item \textbf{Integrar o algoritmo de seleção com o Hacknizer}\\
          Complementar o serviço de \textbf{Seleção} do Hacknizer, adicionando um componente que recebe a lista de inscritos para um dado hackathon e é capaz de gerar uma recomendação dos participantes para o evento, aplicando o algoritmo desenvolvido no passo anterior.
        
        \item \textbf{Investigar possíveis melhorias no algoritmo de seleção para o HackathonUSP}\\
          Estabelecer um critério de seleção que minimize o grau de desistência ao passo que maximize a diversidade (gênero, ano de ingresso, curso, unidade) entre os participantes do evento.
    \end{itemize}
    
%%%%%%%%%%%%%%%%%%%%%%%%%%%%%%%%%%%%%%%%%%%%%%%%%%%%%%%%%%%%%%%%%%%%%%%%
    
\section{Metodologia}
%%%%%%%%%%%%%%%%%%%%%%%%%%%%%%%%%%%%%%%%%%%%%%%%%%%%%%%%%%%%%%%%%%%%%%%%
      
      \begin{itemize}
          \item \textbf{Desenvolvimento do algoritmo} \\
          Para o desenvolvimento do algoritmo de seleção e estudo do conjunto de dados disponíveis, a linguagem \textit{Python} será utilizada devido à disponibilidade da biblioteca \href{http://scikit-learn.org}{\textit{scikit-learn}} que contém a implementação de diversos algoritmos para aprendizagem de máquina, e de \href{http://jupyter.org/}{\textit{Jupyter Notebooks}}, uma aplicação web de código aberto que permite a criação e compartilhamento de documentos contendo código \textit{Python} e visualizações de dados. Em particular, será utilizado um modelo de \textbf{florestas aleatórias}, um classificador que se baseia em múltiplas \textbf{árvores de decisão} aplicadas a um problema de classificação de forma que cada árvore classificadora é independente e contribui com o mesmo peso na classificação final \cite{Breiman2001RandomForests}. Variando os parâmetros considerados, será analisado como o algoritmo se comporta e qual seria o melhor conjunto de parâmetros para os nossos objetivos.
          
          \item \textbf{Integração com o Hacknizer} \\
          Para integrar o algoritmo desenvolvido com o Hacknizer, será desenvolvido um componente que fara parte do serviço de \textbf{Seleção} da plataforma. Esse componente será construído como uma API usando \href{http://flask.pocoo.org/}{\textit{Flask}}, um \textit{microframework} para desenvolvimento de APIs em \textit{Python}, e \href{https://graphene-python.org/}{\textit{Graphene}}, uma implementação em \textit{Python} da especificação \textit{GraphQL} de APIs. Essa API também será conteinerizada via \textit{Docker} e sua implantação poderá ser feito junto com os demais serviços do Hacknizer.
          A especificação \textit{GraphQL} foi escolhida para a API porque todas as APIs dos outros serviços do Hacknizer a utilizam, ao passo que \textit{Python} com \textit{Flask} se integrará com a parte envolvendo aprendizagem de máquina.
      \end{itemize}
%%%%%%%%%%%%%%%%%%%%%%%%%%%%%%%%%%%%%%%%%%%%%%%%%%%%%%%%%%%%%%%%%%%%%%
    
\section{Plano de Trabalho}
%%%%%%%%%%%%%%%%%%%%%%%%%%%%%%%%%%%%%%%%%%%%%%%%%%%%%%%%%%%%%%%%%%%%%%%%
  
  Os seguintes passos serão necessários para cumprir com os objetivos propostos:

  \begin{itemize}
    \item \textbf{Limpeza e organização dos dados a serem analisados (25 horas)} \\
        Nessa primeira fase da pesquisa, com previsão de durar por volta de um mês e meio, será feita a coleta e agregação do conjunto de dados utilizado para a pesquisa, além da normalização desses dados e a geração de visualizações para primeiros \textit{insights}. Esses dados serão coletados das edições anteriores do HackathonUSP.
    
    \item \textbf{Desenvolvimento do algoritmo (40 horas)} \\
        Na segunda fase da pesquisa, serão definidos os critérios preliminares de decisão para o algoritmo de seleção e a comparação de diferentes modelos segundo critérios medidos sobre dados de validação.
          
    \item \textbf{Desenvolvimento do serviço (25 horas)} \\
        Após decidir o algoritmo a ser utilizado, será desenvolvido um novo componente do serviço de seleção do Hacknizer, utilizando Flask para criar uma API GraphQL  que receberá uma lista de inscritos para uma edição de um determinado hackathon e, aplicando o algoritmo desenvolvido, poderá gerar uma lista sugerindo os participantes para tal hackathon. 
    
    \item \textbf{Feedback (10 horas)} \\
        O HackathonUSP 2018.2, segunda edição da série neste ano, irá ocorrer nos dias 9 e 10 de novembro. O serviço desenvolvido será utilizado para obter uma sugestão dos participantes a serem chamados para o evento. Com isso, será possível obter feedback do algoritmo criado, na perspectiva de organizador do evento e também dos participantes. Esse feedback será coletado com os participantes pós-evento, enquanto a perspectiva dos organizadores virá de dentro do grupo \textit{USPCodeLab} em uma reunião pós-evento. Com esses resultados, será elaborado o relatório final da pesquisa para este semestre.

  \end{itemize}
  
%%%%%%%%%%%%%%%%%%%%%%%%%%%%%%%%%%%%%%%%%%%%%%%%%%%%%%%%%%%%%%%%%%%%%%%%

\section{Método de acompanhamento}
%%%%%%%%%%%%%%%%%%%%%%%%%%%%%%%%%%%%%%%%%%%%%%%%%%%%%%%%%%%%%%%%%%%%%%%%

    Na página https://victorhmp.github.io/mac0215-blog/ divulgarei semanalmente as discussões realizadas com meus supervisores e os avanços na pesquisa. Além da página, o código sendo desenvolvido estará no repositório do Hacknizer no GitLab: https://gitlab.com/uspcodelab/projects/hacknizer contento todo o código sendo desenvolvido.
    
    Meus supervisores serão: Professor Alfredo Goldman, que pode ser contatado pelo email: gold@ime.usp.br, e o aluno de mestrado Renato Cordeiro Ferreira, que pode ser contatado pelo email: renatocf@gmail.com.

%%%%%%%%%%%%%%%%%%%%%%%%%%%%%%%%%%%%%%%%%%%%%%%%%%%%%%%%%%%%%%%%%%%%%%%%

\bibliographystyle{alpha}
\bibliography{mendeley}

\end{document}