\documentclass[10pt,twoside,a4paper]{article}
% ^- openany - open new pages in odd/even page

%packages
\usepackage[T1]{fontenc}
\usepackage[utf8]{inputenc}      % Encoding
\usepackage[portuguese]{babel}   % Correção
%\usepackage{caption}             % Legendas
\usepackage{enumerate}

% Matemática
\usepackage{amsmath}             % Matemática
\usepackage{amsthm, amssymb}     % Matemática
\newtheorem*{def*}{Definição}
\newtheorem*{invariant}{Invariante}

% Gráficos
\usepackage[usenames,dvipsnames]{color}  % Cores
\usepackage[pdftex]{graphicx}   % usamos arquivos pdf/png como figura
\usepackage[usenames,svgnames,dvipsnames]{xcolor}

% Desenhos
\usepackage{tikz}
\usepgfmodule{decorations}
\usetikzlibrary{patterns}
\usetikzlibrary{decorations.shapes}
\usetikzlibrary{shapes.geometric}
\usetikzlibrary{decorations.text}
\usetikzlibrary{positioning} % Adjust grid size

% Código-fonte
\usepackage[noend]{algpseudocode}
\usepackage{algorithm}

% Configurações da página
\usepackage{fancyhdr}           % header & footer
\usepackage{float}
\usepackage{setspace}           % espaçamento flexível
\usepackage{indentfirst}        % Identa primeiro parágrafo
\usepackage{makeidx}
\usepackage[nottoc]{tocbibind}  % acrescentamos a  bibliografia/indice/
                                % conteudo no Table of Contents
                                
% Fontes
%\usepackage{helvet}
\renewcommand{\familydefault}{\sfdefault}
\usepackage{type1cm}            % fontes realmente escaláveis
\usepackage{titletoc}
\usepackage{pdflscape}          % Páginas em paisagem
\usepackage{pdfpages}

% Fontes e margens
\usepackage[fixlanguage]{babelbib}
\usepackage[font=small,format=plain,labelfont=bf,up,textfont=it,up]{caption}
\usepackage[a4paper,top=3.0cm,bottom=3.0cm,left=2.0cm,right=2.0cm]{geometry}

% Referências e citações
\usepackage[
    pdftex,
    breaklinks,
    plainpages=false,
    pdfpagelabels,
    pagebackref,
    colorlinks=true,
    citecolor=DarkGreen,
    linkcolor=DarkBlue,
    urlcolor=DarkRed,
    filecolor=green,
    bookmarksopen=true
]{hyperref} 
\usepackage[all]{hypcap} % Soluciona o problema com o hyperref e capitulos
\usepackage[round,sort,nonamebreak]{natbib} % Citação bibliográfica plainnat-ime
%\bibpunct{(}{)}{;}{a}{\hspace{-0.7ex},}{,}  % Estilo de citação
\bibpunct{(}{)}{;}{a}{,}{,}

% Info
\title{MAC0215 - Proposta}
\author{
  \begin{tabular}{rl}
    Aluno:      & Victor Hugo Miranda Pinto \\
    Supervisor: & Alfredo Goldman vel Lebman \\
    Co-supervisor: & Renato Cordeiro
  \end{tabular}
}
\date{2º Semestre de 2018}

\graphicspath{ {./img/} }
 
%%%%%%%%%%%%%%%%%%%%%%%%%%%%%%%%%%%%%%%%%%%%%%%%%%%%%%%%%%%%%%%%%%%%%%%%
\begin{document}

\maketitle
%%%%%%%%%%%%%%%%%%%%%%%%%%%%%%%%%%%%%%%%%%%%%%%%%%%%%%%%%%%%%%%%%%%%%%%%

\section{Resumo}
%%%%%%%%%%%%%%%%%%%%%%%%%%%%%%%%%%%%%%%%%%%%%%%%%%%%%%%%%%%%%%%%%%%%%%%%
  Será escrito por último.

\section{Introdução}
%%%%%%%%%%%%%%%%%%%%%%%%%%%%%%%%%%%%%%%%%%%%%%%%%%%%%%%%%%%%%%%%%%%%%%%%

  Introduzir tópicos necessários para o entendimento do trabalho.

  \subsection{Hackathons}
  %%%%%%%%%%%%%%%%%%%%%%%%%%%%%%%%%%%%%%%%%%%%%%%%%%%%%%%%%%%%%%%%%%%%%%
  
    Introduzir o conceito de Hackathon, de maneira mais geral e os tipos de Hackathons.

    
  \subsection{HackathonUSP}
  %%%%%%%%%%%%%%%%%%%%%%%%%%%%%%%%%%%%%%%%%%%%%%%%%%%%%%%%%%%%%%%%%%%%%%

    Introduzir a ideia e a visão por trás do HackathonUSP
    
  \subsection{Seleção de participantes}
  %%%%%%%%%%%%%%%%%%%%%%%%%%%%%%%%%%%%%%%%%%%%%%%%%%%%%%%%%%%%%%%%%%%%%%
    
    % Introduzir e ressaltar como a seleção de participantes é importante e como ela é realizada atualmente. Esse seria o problema que o resultado final dessa pesquisa iria resolver.
    Um dos elementos mais importantes e decisivos para o sucesso do HackathonUSP é a seleção dos seus participantes, pois é através dela que podemos reforçar a visão que nós do USPCodeLab temos sobre o evento no que diz respeito à tornar o HackathonUSP acessível para todos os alunos da universidade.
    Atualmente, seguimos um algoritmo de seleção que pode ser muito melhorado:
    
    \begin{itemize}
      \item \textbf{Definimos o número (M) de participantes no evento}: Definimos o número de participantes que esperamos receber baseado na infraestrutura disponível para o evento e no número que consideramos razoável para manter a qualidade do evento.

      \item \textbf{Recebemos as N inscrições para o HackathonUSP}: Abrimos as inscrições para os participantes pelo nosso site, em uma data e horário divulgado previamente.
      
      \item \textbf{Selecionamos os M primeiros inscritos}: são selecionados os primeiros M inscritos, por ondem de inscrição.
      
      \item \textbf{Consideramos a taxa de desistência (D)}: selecionamos mais inscritos para obter M participantes efetivos contando com um percentual D de desistência (em geral, 150\% de M para D igual a 33\%). Para essa etapa da seleção, levamos em consideração: completamento de times, selecionando um inscrito que participa de um time cujos demais integrantes foram selecionados nos primeiros M; paridade de gêneros, selecionando mulheres que não tenham times para equilibrar a proporção de gênero entre os selecionados (a área da computação conta com um percentual pequeno de mulheres comparativamente ao de homens); desempenho em edições anteriores, como último critério de completamento, podemos selecionar times cujos membros tiveram destaque em edições anteriores mesmo que nenhum deles estivesse entre os N primeiros
    \end{itemize}
    
    Com a minha pesquisa, poderíamos ter um processo de seleção mais elaborado, visando aumentar a qualidade do evento.
    

  \subsection{Eventual introdução aos algoritmos para análise dos dados}
  %%%%%%%%%%%%%%%%%%%%%%%%%%%%%%%%%%%%%%%%%%%%%%%%%%%%%%%%%%%%%%%%%%%%%%
    

\section{Objetivos}
%%%%%%%%%%%%%%%%%%%%%%%%%%%%%%%%%%%%%%%%%%%%%%%%%%%%%%%%%%%%%%%%%%%%%%%%

    % Ressaltar os objetivos dessa pesquisa e o que se espera como resultado final
    Nessa pesquisa, tenho como objetivo:
    \begin{itemize}
      \item \textbf{Estabelecer um novo critério de seleção para o  HackathonUSP}\\
        Com essa pesquisa, espero estabelecer um critério de seleção que minimize o grau de desistência ao passo que maximize a diversidade (gênero, ano de ingresso, curso, unidade) entre os participantes do evento.

      \item \textbf{Criar um serviço automático de seleção de participantes}\\
        Após o desenvolvimento do algoritmo de seleção, desenvolver um serviço que faça a sugestão automática de participantes para o HackathonUSP pautada nos critérios citados acima.
    \end{itemize}
    
\section{Plano de Trabalho}
  
  Para atingir os objetivos propostos, planejo seguir as seguintes etapas:

  \begin{itemize}
    \item \textbf{Preparo dos dados a serem analisados} \\
        Nessa primeira fase da pesquisa, com previsão de durar por volta de um mês e meio, será feita a coleta e agregação do conjunto de dados que será utilizado para a pesquisa, além da normalização desses dados e a geração de visualizações para primeiros insights sobre os dados que tenho a disposição. Esses dados serão obtidos das inscrições nas edições passadas do HackathonUSP.
    
    \item \textbf{Desenvolvimento do algoritmo} \\
        Na segunda fase da pesquisa, seriam definidos os critérios preliminar de decisão para o algoritmo de seleção e a comparação de diferentes modelos segundo critérios de acurácia. Assim espero obter o algoritmo que será utilizado no primeiro teste da pesquisa em um conjunto diferente do conjunto de teste.
          
    \item \textbf{Desenvolvimento do serviço} \\
        Após decidir o algoritmo a ser utilizado sobre os futuros dados de teste, será iniciado o desenvolvimento de um módulo integrante da plataforma Hacknizer, desenvolvida pelo USPCodeLab, utilizando Flask para criar uma GraphQL API que receberá uma lista de inscritos para uma edição de um determinado Hackathon e, aplicando o algoritmo desenvolvido, poderá gerar uma lista sugerindo os participantes para tal Hackathon. 
    
    \item \textbf{Feedback} \\
        O segundo HackathonUSP de 2018 irá ocorrer no dia 02 de novembro. Planejo utilizar o serviço desenvolvido para obter uma sugestão dos participantes a serem chamados para o evento. Com isso, poderia obter feedback de toda a pesquisa realizada e do algoritmo criado, na perspectiva de organizador do evento e também dos participantes. Com esses resultados, será elaborado o resultado final da pesquisa para esse semestre. Ressaltando que o serviço gerado e o algoritmo desenvolvido continuará a ser aperfeiçoado.

  \end{itemize}

\section{Método de acompanhamento}

    Na página https://victorhmp.github.io/mac0215-blog/ divulgarei semanalmente as discussões realizadas com meus supervisores e os avanços na pesquisa. Além da página, manterei um repositório no GitHub em \\
https://github.com/victorhmp/hackathon-selection contento todo o código sendo desenvolvido.

\end{document}